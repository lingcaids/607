\documentclass{beamer}

\usepackage{graphicx, algpseudocode, algorithm}

\makeatletter
\renewcommand{\fnum@algorithm}{\fname@algorithm}
\makeatother

\usepackage{graphicx, amsmath, amssymb}

% \input{../../../../texmacros/commands.tex}

\newcommand{\vu}{\boldsymbol{u}}
\newcommand{\vv}{\boldsymbol{v}}
\newcommand{\vw}{\boldsymbol{w}}
\newcommand{\vx}{\boldsymbol{x}}
\newcommand{\vzero}{\boldsymbol{0}}

\newcommand{\RR}{\mathbb{R}}

\usetheme{Madrid}

\DeclareMathOperator{\RSS}{RSS}
\DeclareMathOperator{\sign}{sign}
\DeclareMathOperator*{\Argmin}{argmin}
\DeclareMathOperator*{\Argmax}{argmax}

\begin{document}

\setlength{\parskip}{1em}
\begin{frame}
    \title{Linear separation and feature maps}
    \date{DATA 607 --- Session 5 --- 11/03/2019}
    \maketitle
\end{frame}

\begin{frame}{Lines in $\RR^2$: Direction vectors}
    A \underline{line in $\RR^2$} is a set of points of the form
    \[
        L_{\vu, \vv} := \{\vu + t\vv : t\in\RR\},
    \]
    where
    \[
        \vu=\begin{bmatrix}
            u_1\\u_2
        \end{bmatrix},\quad
        \vv=\begin{bmatrix}
            v_1\\v_2
        \end{bmatrix}\in\RR^2,\quad
        \vv\neq \vzero=\begin{bmatrix}
            0\\0
        \end{bmatrix}.
    \]

    Note that $\vu=\vu + 0\vv\in L_{\vu, \vv}$.

    $L_{\vu, \vv}$ is called the line through $\vu$ with \underline{direction vector} $\vv$.

    \begin{itemize}
        \item $\vu'\in L_{\vu, \vv} \Longrightarrow L_{\vu', \vv}=L_{\vu', \vv}$
        \item $\vv'=c\vv,\; c\in\RR,\; c\neq 0 \Longrightarrow L_{\vu, \vv'}=L_{\vu, \vv}$
    \end{itemize}
\end{frame}

\begin{frame}{Lines in $\RR^2$: Half-planes; sides; normal vectors}


    \underline{Dot product}:
    \[
        \vu\cdot\vv = \begin{bmatrix}
            u_1\\u_2
        \end{bmatrix}\cdot\begin{bmatrix}
            v_1\\v_2
        \end{bmatrix} = u_1 v_1 + u_2 v_2\in\RR
    \]
    $\vu$ and $\vv$ are \underline{orthogonal} or \underline{perpendicular}.

    $\vw$ is a \underline{normal vector} to $L_{\vu, \vv}$ if $\vv\cdot\vw = 0$.


\end{frame}

\begin{frame}{Lines in $\RR^2$: Half-planes and sides}
    If you delete a line, $L$, from $\RR^2$, you're left with two \underline{half-planes} called the \underline{sides} of $L$.

    If $\vw$ is a nonzero normal vector to $L_{\vu, \vv}$, then the sets
    \[
        H_{\vu, \vw}^- =
        \{\vx : \vw\cdot(\vx - \vu) < 0\}
        \quad\text{and}\quad
        H_{\vu, \vw}^+ = \{\vx : \vw\cdot(\vx - \vu) > 0\}
    \]
    are the sides of $L_{\vu, \vv}$.

    The normal vector $\vw$, plotted with its tail on $L_{\vu, \vv}$, points into $H_{\vu, \vw}^+$.
\end{frame}

\begin{frame}{Linear separation}
    Let
    \[
        D=\big\{(\vx_1, y_1), (\vx_2, y_2),\ldots, (\vx_n, y_n)\big\}
    \]
    be a dataset, where $\vx_i\in \RR^2$ and $y_i\in \{-1, 1\}$.

    Let
    \[
        D^-=\big\{(\vx_i, y_i)\in D : y_i=-1\big\},
        \quad
        D^+=\big\{(\vx_i, y_i)\in D : y_i=+1\big\}
    \]

    Let $L$ be a line in $\RR^2$. We say that \underline{$L$ separates $D$} if $D^-$ and $D^+$
    are contained in opposite sides of $L$.

    We say that $D$ is \underline{linearly separable} if there is a line $L$ that separates $D$.
\end{frame}

\begin{frame}{}

    \textbf{Not all datasets are linearly separable:} The dataset
    \[
        D := \Big\{\big((0, 0), 0\big), \big((1, 0), 1\big),
        \big((1, 1), 0\big), \big((0, 1), 1\big) \Big\}
    \]
    is not linearly separable.

    \textbf{Linear separators need not be unique:} The linear separators of
    \[
        D := \Big\{\big((0, -1\big), 0), \big((0, 1), 1\big) \Big\}
    \]
    are precisely the lines
    \[
        y=mx+b,\quad b\in (-1, 1).
    \]



\end{frame}

\begin{frame}{Finding linear separators}
    \textbf{Problem:}
    Given a dataset, $D$, find a vector $\vu$ and a nonzero vector $\vw$
    such that
    \[
        D\cap H_{\vu, \vw}^+ = D^+
        \quad\text{and}\quad
        D\cap H_{\vu, \vw}^- = D^-
    \]
    or show that no such $\vu$ and $\vw$ exist.

    We'll begin by analyzing a special case:

    \textbf{Special case:} Given a dataset $D$, find a nonzero vector $\vw$
    such that
    \[
        D\cap H_{\vzero, \vw}^+ = D^+
        \quad\text{and}\quad
        D\cap H_{\vzero, \vw}^- = D^-
    \]
    $L_{\vzero, \vw}$ does \textbf{not} separate $D$ if and only if
    \[
        D^-\cap H_{\vzero, \vw}^+\neq\varnothing
        \quad\text{or}\quad
        D^+\cap H_{\vzero, \vw}^-\neq\varnothing,
    \]
    ...
\end{frame}

\begin{frame}{}

    ...or, equivalently, if and only if
    \[
        \Big(y_i = -1\quad\text{and}\quad \vw\cdot\vx_i >0\Big)
        \quad\text{or}\quad
        \Big(y_i = +1\quad\text{and}\quad \vw\cdot\vx_i <0\Big)
    \]
    for some $i$, or, equivalently, if and only if
    \[
        y_i (\vw\cdot\vx_i) <0
    \]
    for some $i$, or, equivalently, if and only if
    \[
        \min(y_i (\vw\cdot\vx_i), 0) < 0
    \]
    for some $i$, or, equivalently,
    \[
        \sum_i\min(y_i (\vw\cdot\vx_i), 0) < 0,
    \]
    or, equivalently, if and only if
    \[
        \sum_i\max(-y_i (\vw\cdot\vx_i), 0) > 0.
    \]
\end{frame}

\begin{frame}
    View the term
    \[
        L(\vw, \vx_i, y_i) :=\max(-y_i(\vw\cdot\vx_i), 0)
    \]
    as a \textbf{penalty} or \textbf{loss} for $\vx_i$ being misclassified by $\vw$,
    i.e., lying on the wrong side of the line through $\vzero$ normal to $\vw$.

    Assume $\vx_i$ is correctly classified, then
    \[
        \Big(y_i = -1\quad\text{and}\quad \vw\cdot\vx_i <0\Big)
        \quad\text{or}\quad
        \Big(y_i = +1\quad\text{and}\quad \vw\cdot\vx_i >0\Big),
    \]
    in which case $y_i(\vw\cdot\vx_i)>0$ and
    \[
         -y_i(\vw\cdot\vx_i) < 0.
    \]
    Thus, the penalty assessed for $\vx_i$ being misclassified by $\vw$
    is
    \[
        L(\vw, \vx_i, y_i) = \max(-y_i(\vw\cdot\vx_i), 0) = 0,
    \]
    appropriate since, by hypothesis, $\vx_i$ is classified correctly!
\end{frame}

\begin{frame}{}
    Conversely, Assume $\vx_i$ is correctly classified, then
    \[
        \Big(y_i = -1\quad\text{and}\quad \vw\cdot\vx_i >0\Big)
        \quad\text{or}\quad
        \Big(y_i = +1\quad\text{and}\quad \vw\cdot\vx_i <0\Big),
    \]
    in which case $y_i(\vw\cdot\vx_i)<0$ and
    \[
         -y_i(\vw\cdot\vx_i) > 0.
    \]
    Thus, the penalty assessed for $\vx_i$ being misclassified by $\vw$
    is strictly positive:
    \[
        L(\vw, \vx_i, y_i) = \max(-y_i(\vw\cdot\vx_i), 0) > -y_i(\vw\cdot\vx_i).
    \]
    Define the \textbf{cost} associated with $\vw$ by
    \[
        C(D, \vw) = \sum_i L(\vw, \vx_i, y_i)
        = \sum_i \max(-y_i(\vw\cdot\vx_i), 0).
    \]
    Then $L(\vw)$ separates $D$ if and only if \[C(D, \vw)=0.\]
\end{frame}

\begin{frame}{General case}
    \begin{align*}
        L(\vu, \vw, \vx_i, y_i) &=\sum_i \max(-y_i(\vw\cdot(\vx_i - \vu)), 0)\\
        \\
        C(D, \vu, \vw) &= \sum_i L(\vu, \vw, \vx_i, y_i)\\
        &=\sum_i \max(-y_i(\vw\cdot(\vx_i - \vu)), 0)
    \end{align*}

    Find
    \[
        \Argmin_{\vu, \vw} C(D, \vu, \vw).
    \]
\end{frame}

\begin{frame}{The Perceptron Algorithm}
    \[
        D=\big\{(\vx_1, y_1), (\vx_2, y_2),\ldots, (\vx_n, y_n)\big\}
    \]
    Suppose that $D$ is linearly separable, i.e.,
    that there is unit vector a $\vw\in\RR^n$ such that
    \[
        \sign(\vw\cdot \vx_i) = y_i
    \]
    for all $i$, or, equivalently, that
    \[
        y_i(\vw\cdot \vx_i) > 0
    \]
    for all $i$.
\end{frame}

% \begin{frame}{}
%     Define a sequence $\vw_0,\vw_1,\ldots$ of vectors in $\RR^n$ by:
%     \begin{algorithmic}[1]
%         \State $\vw_0 \gets \vzero$
%         \While{blah}
%         blah
%         \EndWhile
%     \end{algorithmic}
    
%     \begin{itemize}
%         \item $\vw_0 = \vzero$
%         \item while $y_i(\vw_k\cdot \vx_i) \leq 0$ for some $i$:
%         \item 
%     \end{itemize}
% \end{frame}

\begin{frame}{}
    Define a sequence $\vw_0,\vw_1,\ldots$ of vectors in $\RR^n$ by:
    \begin{algorithm}[H]
        \caption{The Perceptron Algorithm}
    \begin{algorithmic}[1]
        \State $k \gets 0$
        \State $\vw_0 \gets \vzero$
        \While{$y_i(\vw_k\cdot \vx_i) \leq 0$ for some $i$}
            \State $k \gets k + 1$
            \State $i_k\gets \min\{i : y_i(\vw_{k-1}\cdot \vx_i) \leq 0\}$
            \State $\vw_k \gets \vw_{k-1} + y_{i_k}\vx_{i_k}$
        \EndWhile
    \end{algorithmic}
\end{algorithm}
Note that $i_1=1$.

\textbf{Theorem:} (Rosenblatt, 1957)
The perceptron algorithm terminates after $k<R^2/r^2$ steps.
\end{frame}

\begin{frame}{}
    \textbf{Proof:} Let $k$ be such that $y_{i_k}(\vw_{k-1}\cdot\vx_{i_k})\leq 0$.

    Lower bound on $\|\vw_k\|^2$:

    Set
    \[
        R = \max_i\|\vx_i\|>0.
    \]

    \vspace{-2em}\begin{align*}
        \|\vw_k\|^2 &= \|\vw_{k-1} + y_{i_k}\vx_{i_k}\|^2\\
        &= \|\vw_{k-1}\|^2 + \|\vx_{i_k}\|^2 + 2y_{i_k}(\vw_{k-1}\cdot\vx_{i_k})\\
        &\geq \|\vw_{k-1}\|^2 + \|\vx_{i_k}\|^2\\
        &\geq \|\vw_{k-1}\|^2 + R^2\\
        &\geq \|\vw_{k-2}\|^2 + 2R^2\\
        &\;\;\vdots\\
        &\geq \|\vw_0\|^2 + kR^2\\
        &= kR^2
    \end{align*}
\end{frame}

\begin{frame}{}
    Upper bound on $\|\vw_k\|^2$:
 
    Set
    \[
        r = \min_i |\vw\cdot \vx_i|>0.
    \]

    \vspace{-2em}\begin{align*}
        \vu\cdot \vw_k &= \vu\cdot(\vw_{k-1} + y_{i_k}\vx_{i_k})\\
        &= \vu\cdot \vw_{k-1} + y_{i_k}(\vu\cdot \vx_{i_k})\\
        &\geq \vu\cdot \vw_{k-1} + r\\
        &\geq \vu\cdot \vw_{k-2} + 2r\\
        &\;\;\vdots\\
        &\geq \vu\cdot\vw_0 + kr\\
        &= kr
    \end{align*}

    \vspace{-2em}\[
        kr\leq \vu\cdot\vw_k\leq \|\vu\|\|\vw_k\| = \|\vw_k\|
    \]

    \vspace{-2em}\[
        k^2r^2\leq \|\vw_k\|^2
    \]
\end{frame}

\begin{frame}{}
    \[
        k^2r^2\leq \|\vw_k\|^2\leq kR^2
    \]

    \[
        kr^2\leq \|\vw_k\|^2\leq R^2
    \]

    \[
        k\leq \frac{R^2}{r^2}
    \]
\end{frame}

\begin{frame}{Feature maps}
    What do we do if our data isn't linearly separable?

    Embed your data in a higher dimensional space in which it is linearly separable.

    The higher dimensional space is called \textbf{feature space}
    and the function mapping \textbf{data space} ---
    the ambient space of our data ---
    into this feature space is called a \textbf{feature map}.
    
    
\end{frame}

\begin{frame}
    Consider the linearly inseparable dataset
    \[
        D := \Big\{\big((0, 0), 0\big), \big((1, 0), 1\big),
        \big((1, 1), 0\big), \big((0, 1), 1\big) \Big\}
    \]
    Define a \textbf{feature map} $\phi:\RR^2\to \RR^3$ by
    \[
        \phi(x_1, x_2) = \big(x_1, x_2, (x_1 - x_2)^2\big)
    \]
    Then
    \[
        \phi(D) = \Big\{\big((0, 0, 0), 0\big), \big((1, 0, 1), 1\big),
        \big((1, 1, 0), 0\big), \big((0, 1, 1), 1\big) \Big\}
    \]
    The points in with class label $0$ and $1$ have third coordinates $0$ and $1$, respectively.
\end{frame}

\begin{frame}{}
    The plane $$x_3=\frac12$$ separates the classes.

    The points in data space mapped by $\phi$ into this separating plane are those that satisfy
    \[
        (x_1 - x_2)^2 = \frac12
    \]
    \[
        x_1 - x_2 = \pm \frac12
    \]
    This pair of lines separates the original dataset $D$.

    More examples in the Jupyter notebook.
\end{frame}

\begin{frame}{Where do features come from?}

    Many classification techniques require features to be hand-crafted for
    a given application.

    A more robust approach is to, as much as possible, \emph{learn} the features.
    Neural networks use this approach.
\end{frame}
\end{document}